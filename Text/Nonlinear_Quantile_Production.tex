\documentclass{article}
\usepackage{graphicx}
\usepackage{verbatim}
\usepackage{dcolumn}
\usepackage{array}
\usepackage{mathtools}
\usepackage{float}
\usepackage{booktabs}
\usepackage{cleveref}
\usepackage{siunitx}
\usepackage{enumitem}
\usepackage{bbm}
\usepackage{xcolor}
\usepackage{amsmath}
\usepackage{amsfonts}
\usepackage{amsthm}
\usepackage{amssymb}
\usepackage{booktabs}
\usepackage{caption}
\usepackage{float}
\usepackage{natbib}
\newtheorem{assump}{Assumption}[subsection]
\newtheorem{lemma}{Lemma}[subsection]
\usepackage[toc,page]{appendix}
\pagenumbering{gobble}
\topmargin 0.0cm
\oddsidemargin 0.2cm
\textwidth 16cm
\textheight 21cm
\footskip 1.0cm
\title{A Dynamic Panel Data Framework for Identification and Estimation of Nonlinear Production Functions}
\author{Justin Doty\thanks{Department of Economics, University of Iowa, S321 Pappajohn Business Building, 21 E Market St, Iowa City, IA 52242. Email: \texttt{justin-doty@uiowa.edu}} and Suyong Song\thanks{Department of Economics and Finance, University of Iowa, W360 Pappajohn Business Building, 21 E Market St, Iowa City, IA 52242. Email: \texttt{suyong-song@uiowa.edu}}
}
\date{\vspace{-5ex}}
\begin{document}
\maketitle{} 
\section{Introduction}

Consider a nonlinear model for a firm's gross-output production function (in logs)
\begin{equation}\label{modely}
y_{it}=\sum_{j=1}^{J_{1}}\beta_{j}(\eta_{it})g_{j}(x_{it})+\omega_{it}(\eta_{it})
\end{equation}
where $y_{it}$ is firm $i$'s' output at time $t$ and $x_{it}$ denotes optimal input choices of the firm which include predetermined inputs such as capital and flexible inputs such as labor and other intermediate inputs. The unobserved productivity is denoted by $\omega_{it}$ which is correlated to input choices of the firm at time $t$. We let the output elasticities $\beta$ and productivity $\omega_{it}$ to be functionally dependent on unobserved production shocks $\eta_{i1},\dots, \eta_{iT}$ that are uncorrelated to input choices and productivity at time $t$.\\

Without loss of generality we normalize $\eta_{it}$ to be uniformly distributed on the interval $[0,1]$. This model corresponds to a nonlinear random coefficient model where the outcome $y_{it}$ is monotonic in $\eta_{it}$. In practice we can allow for nonlinear interactions between inputs and unobserved productivity at different quantiles so that marginal effects can be modeled as non-Hick's neutral. For empirical simplicity we assume separability in the unobserved productivity. The function $g$ is an unknown nonlinear function.

In this model, heterogeneity in production technology across firms is driven by the rank of the unobserved production shocks $\eta_{it}$. In the following section we discuss how in certain specification, this is a valid approach to modeling heterogeneous input choices by connecting to the literature on production uncertainty and the firm's profit maximization problem.

We specify the productivity process as
\begin{equation}\label{modelw}
\omega_{it}=\sum_{j=1}^{J_{2}}\rho_{j}(\xi_{it})h_{j}(\omega_{it-1})
\end{equation}

where $\xi_{i1},\dots, \xi_{iT}$ are independent uniform random variables which represent innovation shocks to productivity. We assume $\omega_{it}$ is monotonic in $\xi_{it}$ We let $h$ be another unknown nonlinear function that allows the persistence in productivity in firms to be nonlinear across different quantiles.

\section{Identification}

To ease notation, let $X_{t}=\{Y_{t}, L_{t}, \iota_{t}\}$ and $W_{t}=\{K_{it}, I_{t}\}$. Our goal is identification of the Markov law of motion $f_{X_{t}, W_{t}, \omega_{t}|X_{t-1}, W_{t-1}, \omega_{t-1}}$ which we assume to be stationary. We assume the researcher observes a panel dataset consisting of i.i.d observations of firm output and input choices with the number of time periods $T\geq 4$ for a large number of firms. We introduce the following assumptions that simplify the expression for the law of motion. 

\begin{assump}(Production Dynamics)\label{markovfeedback}
~
    \begin{enumerate}[label=\alph*)]
        \item \textit{Non-dynamic output and inputs:} $f_{X_{t}, W_{t}, \omega_{t}|X_{t-1}, W_{t-1}, \omega_{t-1}}=f_{X_{t}, W_{t}, \omega_{t}|W_{t-1}, \omega_{t-1}}$
        \item \textit{First-order Markov:} $f_{X_{t}, W_{t}, \omega_{t}|X_{t-1}, W_{t-1}, \omega_{t-1}, \mathcal{I}_{<t-1}}=f_{X_{t}, W_{t}, \omega_{t}|X_{t-1}, W_{t-1}, \omega_{t-1}}$, where $\mathcal{I}_{<t-1}$ is the firm's information set up to time $t-1$
        \item \textit{Limited Feedback:} $f_{W_{t}|W_{t-1}, \omega_{t}, \omega_{t-1}}=f_{W_{t}|W_{t-1}, \omega_{t}}$ 
    \end{enumerate}
\end{assump}

Assumption \eqref{markovfeedback}(a) states that input decisions for labor and material inputs maximize profits in the current period and do not have any dynamic implications. This can be relaxed to include cases where labor has dynamic effects due to hiring/firing costs. This assumption also does not allow for dynamic effects of output, which is standard in the production function literature.\\ 

Using Assumption \eqref{markovfeedback} the Markov law of motion can be factored into:
\begin{equation} \label{markovfactor}
    \begin{split}
        &f(X_{t}, W_{t}, \omega_{t}|X_{t-1}, W_{t-1}, \omega_{t-1}, \mathcal{I}_{<t-1})=f(X_{t}, W_{t}, \omega_{t}|W_{t-1}, \omega_{t-1})\\
        &=f(X_{t}|W_{t}, \omega_{t}, W_{t-1}, \omega_{t-1})f(W_{t}|\omega_{t}, W_{t-1}, \omega_{t-1})f(\omega_{t}|W_{t-1}, \omega_{t-1})
    \end{split}
\end{equation} 
We can simplify the first density on the last line of the equation \eqref{markovfactor} as
\begin{equation} \label{1stdensity}
\begin{split}
f(X_{t}|W_{t}, \omega_{t}, W_{t-1}, \omega_{t-1})&=f(Y_{t}|W_{t}, \omega_{t}, W_{t-1}, \omega_{t-1})f(L_{t}|W_{t}, \omega_{t}, W_{t-1}, \omega_{t-1})(\iota_{t}|W_{t}, \omega_{t}, W_{t-1}, \omega_{t-1})\\
&=f(Y_{t}|L_{t}, \iota_{t}, K_{t}, \omega_{t})f(L_{t}|K_{t}, \omega_{t})f(\iota_{t}|K_{t}, \omega_{t})
\end{split}
\end{equation}
Furthermore, the second density on the last line of the equation \eqref{markovfactor} becomes
\begin{equation} \label{2nddensity}
    \begin{split}
        f(W_{t}|\omega_{t}, W_{t-1}, \omega_{t-1})&=f(W_{t}|W_{t-1}, \omega_{t})=f(I_{t}, K_{t}|I_{t-1}, K_{t-1}, \omega_{t})\\
        &=f(I_{t}|K_{t}, I_{t-1}, K_{t-1}, \omega_{t})f(K_{t}|I_{t-1}, K_{t-1}, \omega_{t})\\
        &=f(I_{t}|K_{t}, \omega_{t})f(K_{t}|I_{t-1}, K_{t-1}, \omega_{t})
    \end{split}
\end{equation} 

To show that both Assumption \eqref{markovfeedback}(b-c) are satisfied, we introduce a dynamic model of firm investment that is a slight modification of \cite{Ericson1995}. We begin by providing the evolution process for $\omega_{t}\in \mathbbm{R}$ which is given by:

\begin{equation} \label{ar1}
\omega_{t}=g(\omega_{t-1}, \xi_{t}),
\end{equation}
where the function $g(\cdot, \xi_{t})$ is strictly increasing in the iid innovation shock, $\xi_{t}\in \mathbbm{R}$. Note that equation \eqref{ar1} implies $f_{\omega_{t}|W_{t-1}, \omega_{t-1}}=f_{\omega_{t}|\omega_{t-1}}$ which implies that productivity evolves exogenously. This can be relaxed when we consider productivity enhancing activities such as R\&D similar to \cite{Doraszelski2013}.\\

Capital accumulates according to the following process:

\begin{equation} \label{kaccum}
K_{t}=\kappa_{t}(K_{t-1}, I_{t-1}, \upsilon_{t}),
\end{equation}
where the function $\kappa$ is strictly increasing in its last argument and $\upsilon_{t}$ denotes an iid shock independent of the other arguments. A special case of equation \eqref{kaccum} is the usual capital accumulation law $K_{t}=(1-\delta)K_{t-1}+I_{t-1}+\upsilon_{t}$. Here $\upsilon_{t}$ are other factors that affect the capital accumulation process \textcolor{red}{need better examples of these}.

In each period, a firm chooses investment to maximize its discounted future profits:
\begin{equation} \label{valuefn}
I_{t}=I^{*}(K_{t}, \omega_{t}, \zeta_{t})=\underset{I_{t}\geq 0}{\operatorname{argmax}}\Bigg[\Pi_{t}(K_{t}, \omega_{t}, \zeta_{t})-c(I_{t})+\beta\mathbbm{E}\big[V_{t+1}(K_{t+1}, \omega_{t+1}, \zeta_{t+1})|\mathcal{I}_{t}\big]\Bigg],
\end{equation}
where $\pi_{t}(\cdot)$ is current period profits as a function of the state variables, $c(I_{t})$ is the cost of current investment and $\beta$ is the firm's discount factor. We introduce an additional state variable $\zeta_{t}$ which could represent other factors that affect firm's investment demand such as capital utilization rates or capital specific technological changes that are iid over time. The optimal investment policy is monotonically increasing in both $\omega_{t}$ and $\zeta_{t}$. This is given by the Bellman equation in \eqref{valuefn} and as long as $\frac{\partial \Pi(K_{t}, \omega_{t}, \zeta_{t})}{\partial K_{t}}$ is increasing in $\zeta_{t}$. A detailed proof is given in the Appendix which follows \cite{Pakesa}. Without loss of generality, we normalize $\zeta_{t}\sim U[0,1]$. In a later section we show how to modify identification and estimation strategies to cases when investment is censored or there is selection bias due to endogenous entry/exit which can be the case in certain industries. Note that the restrictions on the capital accumulation process in \eqref{kaccum} and the investment problem in \eqref{valuefn} satisfy the Limited Feedback condition in Assumption \eqref{markovfeedback}(c).

The specifications for the static inputs, labor and intermediate inputs such as materials, fuels, and electricity, are much easier to state. We let optimal labor demand be given by:
\begin{equation} \label{labordemand}
l_{t}=\ell(k_{t}, \omega_{t}, \epsilon_{t})
\end{equation}
where the function, $\ell(\cdot, \cdot, \epsilon_{t})$ is strictly increasing in $\epsilon_{t}$ which is assumed to be independent of the other arguments. We normalize this to be standard uniform each period. We follow a similar model for the intermediate inputs:
\begin{equation} \label{intdemand}
\iota_{t}=\iota_{t}(k_{t}, \omega_{t}, \varepsilon_{t})
\end{equation}
where the function, $\iota(\cdot, \cdot, \varepsilon_{t})$ is strictly increasing in $\varepsilon_{t}$ which is assumed to be independent of the other arguments. We normalize this to be standard uniform each period.

In order to factorize the density in equation \eqref{1stdensity} we need the following assumptions:
\begin{assump}(Conditional Independence)\label{inputindependence}
~
    \begin{enumerate}[label=\alph*)]
        \item The production shock $\eta_{t}$, labor shock $\epsilon_{t}$ and intermediate input shock $\varepsilon_{t}$ are mutually independent conditional on $(W_{t}, \omega_{t}, W_{t-1}, \omega_{t-1})$
        \item The production shock $\eta_{t}$ is independent of $\zeta_{t}$ conditional on $(L_{t}, \iota_{t}, K_{t}, \omega_{t})$
        \item $\epsilon_{t}$ and $\varepsilon_{t}$ are independent of $\zeta_{t}$ conditional on $(K_{t}, \omega_{t})$
    \end{enumerate}
\end{assump}
Assumption \eqref{inputindependence} is similar to the mutual independence assumptions made by \cite{Hu2019}. In their paper, they provide interpretations of these errors that are likely to satisfy the conditional independence restrictions. For example, this assumption is likely to be valid if we interpret each of these shocks as optimization errors. Unlike \cite{Hu2019} we cannot interpret these shocks as unobserved idiosyncratic cost shocks since this would imply that input demand functions, are decreasing in cost shocks, which violates the monotonically increasing assumption we need for representing these functions by their conditional quantiles.\\

We outline the identification procedure similar to \cite{Hu2012}. First we define a function
\begin{equation}\label{dimreduce}
V_{t}\equiv h_{t}(\{X_{t}, W_{t}\}),
\end{equation}
 where the function $h_{t}:\mathbbm{R}^{d_{X}+d_{W}}\rightarrow \mathbbm{R}$ is known with $d_{X}$ being the dimension of $X_{t}$ and $d_{W}$ being the dimension of $W_{t}$. Under assumption \eqref{markovfeedback}, we can write:
 \begin{equation}\label{obsdens}
 f_{X_{t+1}, W_{t+1}, X_{t}, W_{t}, X_{t-1}, W_{t-1}, X_{t-2}, W_{t-2}}=\int f_{X_{t+1},W_{t+1}|W_{t},\omega_{t}}f_{X_{t}, W_{t}|W_{t-1},\omega_{t}}f_{X_{t-1}, W_{t-1}, X_{t-2}, W_{t-2}, \omega_{t}}d\omega_{t}
 \end{equation}
 \textcolor{red}{Should this be factored further?}

The linear operator $L_{V_{t-2},\bar{x}_{t-1},\bar{w}_{t-1},\bar{x}_{t},\bar{w}_{t}, V_{t+1}}$ as a mapping from the $\mathcal{L}^{p}$ space of functions of $V_{t+1}$ to the $\mathcal{L}^{p}$ space of functions of $V_{t-2}$ is defined as

\begin{equation}\label{operator}
    \begin{split}
     (&L_{V_{t-2},\bar{x}_{t-1},\bar{w}_{t-1},\bar{x}_{t},\bar{w}_{t}, V_{t+1}}h)(v_{t-2})\\
     &=\int f_{V_{t-2}, X_{t-1}, W_{t-1}, X_{t}, W_{t}, V_{t+1}}(v_{t-2}, \bar{x}_{t-1}, \bar{w}_{t-1}, \bar{x}_{t}, \bar{w}_{t}, v_{t+1})h(v_{t+1})dv_{t+1}:\\
     &h\in \mathcal{L}^{p}(\mathcal{V}_{t+1}), \bar{x}_{t-1}\in \mathcal{X}_{t-1}, \bar{x}_{t}\in \mathcal{X}_{t}, \bar{w}_{t-1}\in \mathcal{W}_{t-1}, \bar{w}_{t}\in \mathcal{W}_{t},
    \end{split}
\end{equation}
where $\mathcal{V}_{t}, \mathcal{X}_{t-1}, \mathcal{X}_{t}, \mathcal{W}_{t-1}$ and $\mathcal{W}_{t}$ are the supports of $V_{t}, X_{t-1}, X_{t}, W_{t-1}$ and $W_{t}$ respectively. We also define the diagonal operator as:
\begin{equation}\label{diagonal}
    \begin{split}
        (D_{\bar{x}_{t}, \bar{w}_{t}|\bar{x}_{t-1}, \bar{w}_{t-1}, \omega_{t}}h)(\omega_{t})&=f_{X_{t}, W_{t}|, W_{t-1}, \omega_{t}}(\bar{x}_{t}, \bar{w}_{t}|\bar{w}_{t-1}, \omega_{t})h(\omega_{t})\\
        &h\in \mathcal{L}^{p}(\Omega_{t}), \bar{x}_{t}\in \mathcal{X}_{t}, \bar{w}_{t}\in \mathcal{W}_{t}, \bar{w}_{t-1}\in \mathcal{W}_{t-1}
    \end{split}
\end{equation}
We provide a few lemmas that allow us to represent the observed density $f_{X_{t+1}, W_{t+1}, X_{t}, W_{t}, X_{t-1}, W_{t-1}, X_{t-2}, W_{t-2}}$ and the Markov law of motion $f_{X_{t}, W_{t}, \omega_{t}|X_{t-1}, W_{t-1}, \omega_{t-1}}$ using operator notation.

\section{Estimation Strategy}
Our estimation strategy does not follow directly from the identification conditions discussed earlier in this paper, but on a set of restrictions discussed in the appendix. Our empirical specification for the Markovian transitions of productivity, output, and capital evolution closely resemble \cite{Arellano2017}. 
\subsection{Persistent Productivity}
Let $age_{it}$ denote the age of firm $i$ at time $t$, we specify productivity to transition according to:
\begin{equation}\label{omegamodel}
Q_{t}(\omega_{it-1}, \tau)=Q(\omega_{it-1}, age_{it}, \tau)=\rho_{0}(\tau)+\rho_{\omega}(\tau)\omega_{it-1}+\rho_{a}(\tau)age_{it}
\end{equation}

\noindent The quantile function for $\omega_{i1}$ is specified in a similar way
\begin{equation}
\label{omega1model}
Q_{\omega_{1}}(age_{i1}, \tau)=\rho_{01}(\tau)+\rho_{a1}(\tau)age_{i1}
\end{equation}

\subsection{Output}
We specify the output equation as follows:
\begin{equation}\label{ymodel}
\begin{split}
Q_{t}(x_{it}, \omega_{it}, \tau)&=Q(x_{it}, age_{it}, \omega_{it}, \tau)=\\
&\beta_{0}(\tau)+\beta_{\omega}(\tau)\omega_{it}+\beta_{a}(\tau)age_{it}+\beta_{l}(\tau)l_{it}+\beta_{k}(\tau)k_{it}+\beta_{m}(\tau)m_{it}+\beta_{u}(\tau)u_{it}
\end{split}
\end{equation}
The above representation of the production function is a simple case. One can also consider more flexible specifications that allow for interactions of productivity and input choices thus allowing for non-Hicks neutral technology changes.

\subsection{Investment Demand}
We specify the investment process as the censored variable, $I_{t}=\text{max}\{0, I^{*}_{t}\}$ where $I_{t}^{*}=\iota_{t}(\omega_{it}, k_{it}, \zeta_{it})$. We specify the investment demand equation as:
\begin{equation}\label{imodel}
\iota_{t}(\omega_{it}, k_{it}, \tau)=\iota(\omega_{it}, k_{it}, age_{it}, \tau)=\iota_{\omega}(\tau)\omega_{it}+\iota_{k}(\tau)k_{it}+\iota_{a}(\tau)age_{it}+\iota_{0}(\tau)
\end{equation}
Due to the equivariance of quantiles to monotone transformations we can write the censored investment variable as:
\begin{equation}\label{censormodel}
Q(\omega_{it}, k_{it}, \tau)=\text{max}\{0, \iota_{0}(\tau)+\iota_{\omega}(\tau)\omega_{it}+\iota_{k}(\tau)k_{it}+\iota_{a}(\tau)age_{it}\}
\end{equation}

\subsection{Static Inputs}
Finally, we specify a reduced-form demand function for the static inputs of $z=m$ and $u$ (in logs) as:
\begin{equation}\label{staticmodel}
\mu_{zt+1}(k_{it+1}, \omega_{it+1}, \tau)=\mu_{z}(k_{it+1}, \omega_{it+1}, age_{it+1}, \tau)=\mu_{zk}k_{it+1}+\mu_{z\omega}\omega_{it+1}+\mu_{za}age_{it}+\mu_{z0}(\tau)
\end{equation}

\section{Implementation}
To ease notation, let $\delta$ denote a finite dimensional parameter vector that contains the functions $\rho_{0},\rho_{\omega}, \rho_{a}, \rho_{01}, \rho_{a_{1}}$ and $\theta$ denote the finite dimensional parameter vector that contains the functions $\beta_{0}, \beta_{\omega}, \beta_{a}, \beta_{l}, \beta_{k}, \beta_{m}, \beta_{u}, \iota_{0}, \gamma_{0}, \gamma_{\omega}, \gamma_{a}, \kappa_{0}$ and $\mu_{x0}$. The vector $\theta$ also contains the finite dimensional parameters $\iota_{\omega}, \iota_{k}, \iota_{a}, \kappa_{K}, \kappa_{I}, \kappa_{a}, \mu_{xk}$ and $\mu_{xa}$. We model the functional parameters using \cite{Wei2009} and\cite{Arellano2016}. For example, the function $\rho_{\omega}$ is modeled as a piecewise-polynomial interpolating splines on a grid $[\tau_{1},\tau_{2}], [\tau_{3},\tau_{4}],\dots, [\tau_{J-1},\tau_{J}]$, contained in the unit interval and is constant on $[0, \tau_{1}]$ and $[\tau_{J}, 1)$ The intercept coefficient $\rho_{0}$ is specified as the quantile of an exponential distribution on $(0,\tau_{1}]$ (indexed by $\lambda_{\omega}^{-}$) and $[\tau_{J-1}, 1)$ (indexed by $\lambda_{\omega}^{+}$).\\

The function $\rho_{\omega}$ is modeled as piecewise linear on $[\tau_{1}, \tau_{J}]$.The remaining functional parameters are modeled similarly. We take $J=11$ and $\tau_{j}=\frac{j}{J+1}$. Following \cite{Arellano2017} we parameterize the distribution of $\epsilon$ to be log-normal so we set, for example, $\mu_{z0}(\tau)=\alpha_{z0}+\sigma_{z}\Phi^{-1}(\tau)$ although this can be relaxed. In the following section we outline the model's restrictions and a feasible estimation strategy.

\subsection{Model Restrictions}
Let $\Psi_{\tau}(u)=\tau-\mathbbm{1}\{u<0\}$ denote the first derivative of the quantile check function $\psi_{\tau}(u)=(\tau-\mathbbm{1}\{u<0\})u$. The following conditional moment restrictions hold as an implication of the conditional independence restrictions in section 2. Therefore, we estimate the parameters of interest from the following conditional moment restrictions.

\begin{equation}\label{qmoments}
\mathbbm{E}\Bigg[\Psi_{\tau_{j}}
\begin{pmatrix}
\xi_{it+1} \\
\eta_{it}
\end{pmatrix}
\Big|\omega_{it}, age_{it}, \tilde{z}_{td}\Bigg]=0
\end{equation}
For investment,
\begin{equation}\label{imoment}
\mathbbm{E}\Bigg[\Psi_{\tau_{j}}(\zeta_{it})|\omega_{it}, k_{it}, age_{it}\Bigg]=0,
\end{equation}
and lastly for the static input:
\begin{equation}\label{staticmoment}
\mathbbm{E}\Bigg[\epsilon_{it+1}|\omega_{it},age_{it}, \tilde{z}_{td}\Bigg]=0,
\end{equation}


\textcolor{red}{Why do we need a moment restriction for $\zeta_{it}$? Current productivity enters the investment equation so we need a model for investment. See footnote 23 page 709 in \cite{Arellano2017}}\\

where $\tilde{z}_{td}=I_{t}, l_{t}, k_{t}, m_{t}$, and $u_{t}$. To fix ideas, we first specify how to estimate the conditional moments for persistent productivity and production.\\

\noindent For $t\geq 2$
\begin{equation}\label{omegamoment}
\begin{split}
\mathbbm{E}\Bigg[&\Psi_{\tau_{j}}(\xi_{it+1})\Big|\omega_{it}, age_{it}, \tilde{z}_{td}\Bigg]=\\
&\mathbbm{E}\Bigg[\Psi_{\tau_{j}}(\omega_{it+1}-\bar{\rho}_{0}(\tau_{j})-\bar{\rho}_{\omega}(\tau_{j})\omega_{it}-\bar{\rho}_{a}(\tau_{j})age_{it})\Big|\omega_{it}, age_{it}, \tilde{z}_{td}\Bigg]=0
\end{split}
\end{equation}
\begin{equation}\label{ymoment}
\begin{split}
\mathbbm{E}\Bigg[&\Psi_{\tau_{j}}(\eta_{it})\Big|\omega_{it}, age_{it}, \tilde{z}_{td}\Bigg]=\\
&\mathbbm{E}\Bigg[\Psi_{\tau_{j}}(y_{it}-\bar{\beta}_{0}(\tau_{j})-\bar{\beta}_{\omega}(\tau_{j})\omega_{it}-\bar{\beta}_{a}(\tau_{j})age_{it}-\bar{\beta}_{l}(\tau_{j})l_{it}-\bar{\beta}_{k}(\tau_{j})k_{it}\\
&-\bar{\beta}_{m}(\tau_{j})m_{it}-\bar{\beta}_{u}(\tau_{j})u_{it})\Big|\omega_{it}, age_{it}, \tilde{z}_{td}\Bigg]=0
\end{split}
\end{equation}

Here $\bar{\rho}(\tau_{j})$ and $\bar{\beta}(\tau_{j})$ denote the true values of $\rho(\tau_{j})$ and $\beta(\tau_{j})$ for $j\in\{1,\dots, J\}$. Clearly, estimating the above conditional moment restrictions are infeasible due to the unobserved productivity component. Therefore, we use the following unconditional moment restrictions and posterior distributions for $\omega_{it}$ to integrate out the unobserved productivity. Due to the law of iterated expectations we now have the following integrated moment conditions:

\begin{equation}\label{omegaimc}
\mathbbm{E}\Bigg[\int\Bigg(\Psi_{\tau_{j}}(\omega_{it+1}-\bar{\rho}_{0}(\tau_{j})+\bar{\rho}_{\omega}(\tau_{j})\omega_{it}-\bar{\rho}_{a}(\tau_{j})age_{it})\otimes
\begin{pmatrix}
1 \\
\omega_{it}\\
age_{it}\\
\tilde{z}_{td}
\end{pmatrix}
\Bigg)g_{i}(\omega_{it};\bar{\delta},\bar{\theta})d\omega_{it}\Bigg]=0
\end{equation}
and 
\begin{equation}\label{yimc}
\begin{split}
\mathbbm{E}\Bigg[\int\Bigg(\Psi_{\tau_{j}}(y_{it}&-\bar{\beta}_{0}(\tau_{j})+\bar{\beta}_{\omega}(\tau_{j})\omega_{it}+\bar{\beta}_{a}(\tau_{j})age_{it}+\bar{\beta}_{l}(\tau_{j})l_{it}+\bar{\beta}_{k}(\tau_{j})k_{it}\\
&+\bar{\beta}_{m}(\tau_{j})m_{it}+\bar{\beta}_{u}(\tau_{j})u_{it})\otimes
\begin{pmatrix}
1 \\
\omega_{it}\\
age_{it}\\
\tilde{z}_{td}
\end{pmatrix}
\Bigg)g_{i}(\omega_{it};\bar{\delta},\bar{\theta})d\omega_{it}\Bigg]=0,
\end{split}
\end{equation}

where $\bar{\theta}$ and $\bar{\delta}$ denote the true values of $\theta$ and $\delta$. The posterior distribution is specified as (age omitted for ease of notation):
\begin{equation}\label{posterior}
\begin{split}
g_{i}(\omega_{it};\bar{\delta},\bar{\theta})&=g(\omega_{it}|m_{it+1}, y_{it}, \tilde{z}_{td}; \bar{\delta},\bar{\theta}) \propto\\
&\prod_{t=1}^{T}g(y_{it}|\omega_{it}, l_{it}, k_{it}, m_{it}, u_{it};\bar{\delta},\bar{\theta})\times
\prod_{t=1}^{T}g(I_{it}|\omega_{it}, l_{it}, k_{it}, m_{it}, u_{it};\bar{\delta},\bar{\theta})\times\\
&\prod_{t=1}^{T}g(m_{it+1}|\omega_{it}, l_{it}, k_{it}, m_{it}, u_{it};\bar{\delta},\bar{\theta})\times \prod_{t=2}^{T}g(\omega_{it}|\omega_{it-1};\bar{\delta})\times g(\omega_{i1};\bar{\delta})
\end{split}
\end{equation}

The unconditional moment restrictions implied by \eqref{qmoments} are over-identified. The posterior density in equation \eqref{posterior} is a closed-form expression when using piece-wise linear splines for $(\delta(\cdot), \theta(\cdot))$. Therefore, we use a Generalized Method of Moments (GMM) objective function to estimate $(\delta, \theta)$ The estimation is an Expectation Maximization (EM) algorithm. In \cite{Arellano2016} and \cite{Arellano2017}, the ``M-step'' is performed using quantile regression. In our M-step, we replace quantile regression with GMM since our moment conditions are smooth due to the presence of the integral. Given an initial parameter value $(\hat{\delta}^{0}, \hat{\theta}^{0})$. Iterate on $s=0,1,2,\dots$ in the following two-step procedure until converge to a stationary distribution:

\begin{enumerate}
    \item \textit{Stochastic E-Step}: Draw $M$ values $\omega_{i}^{(m)}=(\omega_{i1}^{(m)}, \omega_{i2}^{(m)},\dots, \omega_{iT}^{(m)})$ from
        \begin{equation*}
        \begin{split}
            g(\omega_{it}&|m_{it+1}, y_{it}, \tilde{z}_{td}; \hat{\delta}^{(s)},\hat{\theta}^{(s)}) \propto\\
            &\prod_{t=1}^{T}g(y_{it}|\omega_{it}, l_{it}, k_{it}, m_{it}, u_{it};\hat{\delta}^{(s)},\hat{\theta}^{(s)})\times
            \prod_{t=1}^{T}g(I_{it}|\omega_{it}, l_{it}, k_{it}, m_{it}, u_{it};\hat{\delta}^{(s)},\hat{\theta}^{(s)})\times\\
            &\prod_{t=1}^{T}g(m_{it+1}|\omega_{it}, l_{it}, k_{it}, m_{it}, u_{it};\hat{\delta}^{(s)},\hat{\theta}^{(s)})\times \prod_{t=2}^{T}g(\omega_{it}|\omega_{it-1};\hat{\delta}^{(s)})\times g(\omega_{i1};\hat{\delta}^{(s)})
            \end{split}
        \end{equation*}
    \item \textit{Maximization Step}: For $j=1,\dots, J$, solve the following GMM objective functions
    \begin{equation*}
    \begin{split}
    \hat{\boldsymbol\rho}(\tau_{j})^{(s+1)}&=\underset{\boldsymbol\rho(\tau_{j})}{\operatorname{argmin}}\,\hat{M}_{n,t,m}(\boldsymbol\rho, \tau_{j})^{'}\hat{W}_{\boldsymbol\rho}\hat{M}_{n,t,m}(\boldsymbol\rho, \tau_{j})\\
    \hat{\boldsymbol\rho_{1}}(\tau_{j})^{(s+1)}&=\underset{\boldsymbol\rho_{1}(\tau_{j})}{\operatorname{argmin}}\,\hat{M}_{n,1,m}(\boldsymbol\rho_{1}, \tau_{j})^{'}\hat{W}_{\boldsymbol\rho_{1}}\hat{M}_{n,1,m}(\boldsymbol\rho_{1}, \tau_{j})\\
    \hat{\boldsymbol\beta}(\tau_{j})^{(s+1)}&=\underset{\boldsymbol\beta(\tau_{j})}{\operatorname{argmin}}\,\hat{M}_{n,t,m}(\boldsymbol\beta, \tau_{j})^{'}\hat{W_{\boldsymbol\beta}}\hat{M}_{n,t,m}(\boldsymbol\beta, \tau_{j})\\
    \end{split}
    \end{equation*}
    where, for example
    \begin{equation*}
    \begin{split}
     \boldsymbol\rho(\tau_{j})&=(\rho_{0}(\tau_{j}), \rho_{\omega}(\tau_{j}), \rho_{a}(\tau_{j}))\\
     \hat{M}_{n,t,m}(\boldsymbol\rho, \tau_{j})&=\sum_{i=1}^{N}\sum_{t=2}^{T}\sum_{m=1}^{M}\Psi_{\tau_{j}}(\xi_{it+1}(\boldsymbol\rho(\tau_{j})))Z_{it}^{(m)}\\
     Z_{it}^{(m)}&=(1, \omega_{it}^{(m)}, age_{it}, \tilde{z}_{td})\\
     \xi_{t+1}(\boldsymbol\rho(\tau_{j}))&=\omega_{it+1}^{(m)}-\rho_{0}(\tau_{j})+\rho_{\omega}(\tau_{j})\omega_{it}^{(m)}-\rho_{a}(\tau_{j})age_{it}
     \end{split}
     \end{equation*}
     and $\hat{W_{\boldsymbol\rho}}$ is a weighting matrix. We estimate each moment restriction separately as computation is less prone to error compared to estimating jointly. \\

     We estimate the investment equation without over-identifying conditions \textcolor{red}{because I do not know how to estimate a censored quantile regression model using GMM}
     \begin{equation}\label{iimc}
     \mathbbm{E}\Bigg[\int\Bigg(\Psi_{\tau_{j}}(I_{it}-\max\{0, \iota_{0}(\tau)+\iota_{\omega}(\tau)\omega_{it}+\iota_{k}(\tau)k_{it}+\iota_{a}(\tau)age_{it}\})\otimes
     \begin{pmatrix}
     1\\
     \omega_{it}\\
     k_{it}\\
     age_{it}
     \end{pmatrix}
     \Bigg) g_{i}(\omega_{it};\bar{\delta},\bar{\theta})d\omega_{it}
     \Bigg]=0
     \end{equation}
     which can be estimated using
     \begin{equation}\label{iobjective}
      \hat{\boldsymbol\iota}(\tau_{j})^{(s+1)}=\underset{\boldsymbol\iota(\tau_{j})}{\operatorname{argmin}}\,\sum_{i=1}^{N}\sum_{t=1}^{T}\sum_{m=1}^{M}\psi_{\tau_{j}}\Bigg(I_{it}-\max\{0, \iota_{0}(\tau)+\iota_{\omega}(\tau)\omega_{it}^{(m)}+\iota_{k}(\tau)k_{it}+\iota_{a}(\tau)age_{it}\Bigg)
     \end{equation}
     this step can be performed using any existing censored quantile regression using the package \textit{quantreg}.\\

     The moment restrictions involving the static input can be written as:
     \begin{equation}\label{staticimc}
     \mathbbm{E}\Bigg[\int(z_{it+1}-\alpha_{z0}-\mu_{zk}k_{it+1}+\mu_{z\omega}\omega_{it+1}-\mu_{za}age_{it})\otimes
     \begin{pmatrix}
     1\\
     \omega_{it}\\
     age_{it}\\
     \tilde{z}_{td}
     \end{pmatrix}
     g_{i}(\omega_{it};\bar{\delta},\bar{\theta})d\omega_{it}
     \Bigg]=0,
     \end{equation}
     which can be estimated using GMM. With estimates ($\hat{\alpha}_{z0}, \hat{\mu}_{k\omega}, \hat{\mu}_{z\omega},\hat{\mu}_{za}$), the variance of the static input demand shock can be found by the following equation:
     \begin{equation}\label{staticvar}
     \hat{\sigma}^{2}_{z}=\frac{1}{T}\sum_{t=1}^{T}\mathbbm{E}\Bigg[\int\Bigg(z_{it+1}-\hat{\alpha}_{z0}-\hat{\mu}_{zk}k_{it+1}+\hat{\mu}_{z\omega}\omega_{it+1}-\hat{\mu}_{za}age_{it})^{2}g_{i}(\omega_{it};\bar{\delta},\bar{\theta})\Bigg)d\omega_{it}\Bigg]
     \end{equation}
\end{enumerate}

\pagebreak
\newpage

\bibliographystyle{ecca.bst}
\bibliography{NL_PF_QR}


\end{document}