\documentclass{beamer}

\mode<presentation> {


%\usetheme{default}
%\usetheme{AnnArbor}
%\usetheme{Antibes} 
%\usetheme{Bergen}
%\usetheme{Berkeley}
%\usetheme{Berlin}
%\usetheme{Boadilla}
%\usetheme{CambridgeUS}
%\usetheme{Copenhagen}
%\usetheme{Darmstadt}
%\usetheme{Dresden}
%\usetheme{Frankfurt}
%\usetheme{Goettingen}
%\usetheme{Hannover}
%\usetheme{Ilmenau}
%\usetheme{JuanLesPins}
%\usetheme{Luebeck}
\usetheme{Madrid}
%\usetheme{Malmoe}
%\usetheme{Marburg}
%\usetheme{Montpellier}
%\usetheme{PaloAlto}
%\usetheme{Pittsburgh}
%\usetheme{Rochester}
%\usetheme{Singapore}
%\usetheme{Szeged}
%\usetheme{Warsaw}
%\usecolortheme{albatross}
%\usecolortheme{beaver}
%\usecolortheme{beetle}
%\usecolortheme{crane}
%\usecolortheme{dolphin}
%\usecolortheme{dove}
%\usecolortheme{fly}
%\usecolortheme{lily}
%\usecolortheme{orchid}
%\usecolortheme{rose}
%\usecolortheme{seagull}
%\usecolortheme{seahorse}
%\usecolortheme{whale}
%\usecolortheme{wolverine}

%\setbeamertemplate{footline} % To remove the footer line in all slides uncomment this line
%\setbeamertemplate{footline}[page number] % To replace the footer line in all slides with a simple slide count uncomment this line

%\setbeamertemplate{navigation symbols}{} % To remove the navigation symbols from the bottom of all slides uncomment this line
}

\usepackage{graphicx} % Allows including images
\usepackage{booktabs} % Allows the use of \toprule, \midrule and \bottomrule in tables
\usepackage{amsmath}
\usepackage{amssymb}
\usepackage{mathrsfs}
\usepackage{caption}
\usepackage{subcaption}
\usepackage{bbm} 
\usepackage{xcolor}
%----------------------------------------------------------------------------------------
%	TITLE PAGE
%----------------------------------------------------------------------------------------

\title[Nonlinear Production Functions]{A Dynamic Panel Data Framework for Identification and Estimation of Nonlinear Production Functions}

\author{Justin Doty} % 
\institute[University of Iowa] % Your institution as it will appear on the bottom of every slide, may be shorthand to save space
\title[Quantile Production Functions]{Heterogeneity in Firms:\\
A Proxy Variable Approach to Quantile Production Functions}

\author{Justin Doty and Suyong Song} % 
\institute[] % Your institution as it will appear on the bottom of every slide, may be shorthand to save space
{
\\  
\medskip % Your email address
}
\date{\today} % Date, can be changed to a custom date

\begin{document}

\begin{frame}
\titlepage % Print the title page as the first slide
\end{frame}

%----------------------------------------------------------------------------------------
%	PRESENTATION SLIDES
%----------------------------------------------------------------------------------------

%------------------------------------------------
\section{First Section} % Sections can be created in order to organize your presentation into discrete blocks, all sections and subsections are automatically printed in the table of contents as an overview of the talk
%------------------------------------------------

\subsection{Subsection Example} % A subsection can be created just before a set of slides with a common theme to further break down your presentation into chunks
\begin{frame}
\frametitle{Introduction}
\begin{itemize} 
\item Consider the simple random-coefficient production function
\begin{equation} \label{pfrc}
    y_{it}=\beta_{0}(\eta_{it})+\beta_{k}(\eta_{it})k_{it}+\beta_{l}(\eta_{it})l_{it}+\beta_{m}(\eta_{it})m_{it} +\omega_{it}
\end{equation}
where $y_{it}$ denotes value-added output, $l_{it}$ denotes labor input for firm $i$ at time $t$, $k_{it}$ denotes capital input, $\omega_{it}$ is unobserved productivity and $\eta_{it}$ denotes an iid shock to production.
\end{itemize}
\end{frame}
%------------------------------------------------------------------------------------


%----------------------------------------------------------------------------------

\end{document}

